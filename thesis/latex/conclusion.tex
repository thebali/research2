\chapter{Conclusion and Future Work}

CHAPTER 8. CONCLUSION

in Chapter 6 results are fairly good despite using less user context, but there might be
other algorithms more suitable for this purpose.
Another conclusion that can be made using Contextual Multi-Armed Bandit Algorithms
in garment-based e-commerce systems, is when the e-commerce system does not make
use of ratings, where otherwise reliable collaborative filtering-based algorithms might
not perform so well (see 6.3.2), Contextual Multi-Armed Bandit Algorithms could work
better because it observes user-context and -behaviour rather than ratings. Although
this thesis is only testing Contextual Multi-Armed Bandit Algorithms using static data
from one garment-based e-commerce system, which does not make use of ratings. To
be certain that equally good results can be achieved frequently, it would be necessary
to test the algorithm on data from more and different systems. Especially significantly
different systems, such as ones that does not involve clothing.
Contextual Multi-Armed Bandit Algorithms work in systems with or without ratings.
As long as there exist recorded data of users and their behaviour in the system, Contextual Multi-Armed Bandit Algorithms should have good chances of providing good
and reliable recommendations in the aspect of user-experience as well as of performing
successful recommendations.
