\chapter{Results}
Matrix Factorization are widely used in recommender systems for e-commerce. In this chapter follows a presentation of the results achieved in this project using our own implementation of The Latent factors algorithms applied to the a dataset from a live e-commerce system. This chapter will focus on the actual results. For evaluation and discussion regarding how good the results were, see Chapter 7.

\section{Recommendation Algorithm Evaluation}
Taking a privacy-preserving approach, recommendations performed by the implementation of the algorithms, the application is presented below. Graphs of each sub category is shown and subsequent comparison table is constructed.

\begin{table}[h]
\centering
\begin{tabular}{| c | c | c | c | c | c |}
\hline
\textbf{Method} & \textbf{Time for Training the Model} & \textbf{RMSE} & \textbf{Recall} & \textbf{Precision}  \\
\hline
Matrix Factorization & 70.49s & 1.1914 & 0.473 & 0.798  \\
\hline
Topic Modelling & 65.33s & 1.8145 & 0.631 & 0.539 \\
\hline
Combined Model & 117.89s  & \textbf{1.0787} & 0.588 & 0.751 \\ 
\hline          
\end{tabular}
\caption{Results Table}
\label{Results Table}
\end{table}

In the above table we can see that the RMS values of the matrix factorization and topic modelling are quite good.
Then the experiment was conducted with two models combined and as we can see that RMSE value of combined models is 1.0787.

This shows a significant increase in the model accuracy, consequently this type of approach can be used to improve the recommendations generated by the system.

\begin{itemize}
    \item Always recommend items based on gender. (Men and women always buy the same items as other men and women, respectively)
    \item Always recommend items based on age. (People at a certain age always buy same items as other people in the same age)
    \item Always recommend items based on domicile. (People from city X always buy the same items as other people from city X)
    \item Baselines have been used before any other extensive evaluation method, something that has been of great help when implementing the application, as these baselines can be seen as evaluation in its most basic form.
    
    \end{itemize}


