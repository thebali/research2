%\centering

% \begin{abstract}
% \vspace*{3em}
% The world has seen many recommender systems online or offline. More or less the working is the same for both. In recommender systems, item-based collborative filtering is a
% \end{abstract}

\begin{center}
\normalsize\textbf{Abstract}\\
\justify{
The topic of Recommender Systems is and have been a hot topic the last century as the
market for e-commerce keeps extending. Basic techniques for recommending popular
items are used on more or less all e-commerce platforms. Many e-commerce based platforms use simple techniques such as "people who bought this also bought that", while others have very complex recommender systems for customised recommendations depending on users profiles. Regardless of what techniques that are being used, companies want to make their customers happy as well as increasing their own profit. Because of this there is a constant demand for smart systems using up-to-date algorithms and techniques to achieve relevant advertisements. This thesis focuses on evaluating the performance of Matrix Factorization Algorithms and Latent Dirichlet Allocation algorithm, for the specific algorithm, not yet fully explored use-area of recommender systems, namely the area of Digital Music and Health Care e-commerce.
The evaluation consists in measuring the performance mostly in terms of successful recommendations, while discussing satisfaction-level of customers. In addition to this we decided to experiment with different privacy-preserving techniques to see how it affects
the performed recommendations. 

Through observing the results of the implemented application we made it possible to identify the better algorithm in Recommender Systems.
}
\end{center}

