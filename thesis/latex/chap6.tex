\chapter{Result}
Multi-Armed Bandit Algorithms are rarely used in recommender systems
for e-commerce. In this chapter follows a presentation of the results achieved
in this project using our own implementation of The Contextual MultiArmed Bandit Algorithm applied to the a dataset from a live e-commerce
system. This chapter will focus on the actual results. For evaluation and discussion
regarding how good the results were, see Chapter 7.

6.1
Recommender System

Taking a privacy-preserving approach, recommendations performed by the implemented
application can be split into three categories or levels of context which can be seen under
6.1.1, 6.1.2 and 6.1.3. In the following series of plots, the results of the implemented
application using Contextual Multi-Armed Bandits are illustrated. Each subsection contains four graphs illustrating different time periods and for each subsection the data is
limited taking a more privacy-preserving approach. The graphs show a percentage of
correct predictions on the y-axis and the total number of predictions made on the x-axis.

6.1.1

Using all available context

The lowest level of privacy consist in using all available context, meaning that the algorithm takes everything from user-submitted data, to observed and learned user-behaviour and -history, into account when predicting purchases. Here follow some graphs with different time aspects, visualising the performance with number of predictions on the x-axis
and correctness in percent on the y-axis using this policy.
Graphs in figure 6.1 show all arms relative to the best arm. The best arm is chosen as the one with best performance posterior of the simulation. Notice that ”All arms” follow the same pattern as the ”Best arm” towards the end. This is due to the fact
32

-------6.1. RECOMMENDER SYSTEM

CHAPTER 6. RESULT

that the algorithm is more or less done with the exploration phase and has started the
exploiting phase, choosing the best arm over and over.

Predictions made of data from 1 week in time

Predictions made of data from 1 month in time
25

Correct predictions (%)

Correct predictions (%)

30

25

20

15
Best Arm
All Arms

10

0

1000

2000

3000

4000

5000

6000

20

15
Best Arm
All Arms

10

7000

0

0.5

Number of predictions
Predictions made of data from 3 months in time

2
×104

25

Correct predictions (\%)

Correct predictions (\%)

1.5

Predictions made of data from 6 months in time

30
25
20
15
10
5

1

Number of predictions

Best Arm
All Arms

0

2

4

6

Number of predictions

8

15

10
Best Arm
All Arms

5

10
×10

20

4

0

0.5

1

Number of predictions

Figure 6.1: Results using all available context

33

1.5

2
×105

6.1. RECOMMENDER SYSTEM

6.1.2

CHAPTER 6. RESULT

Using only user-submitted data

The middle level of privacy consist in only using data submitted by the users of a system. Data that they agreed to submit upon registering an account or purchase in the
system. Recommendations are done using only this data, without having the algorithm
train on observed user behaviour in the system. Here follow some graphs with different
time aspects, visualising the performance with number of predictions on the x-axis and
correctness in percent on the y-axis using this policy.

Predictions made of data from 1 week in time

Predictions made of data from 1 month in time
14

Correct predictions (%)

Correct predictions (%)

15

10

5
Best Arm
All Arms

0

0

1000

2000

3000

4000

5000

6000

12
10
8
6
4
2

7000

Best Arm
All Arms

0

0.5

Number of predictions
Predictions made of data from 3 months in time

2
×104

15

Correct predictions (%)

Correct predictions (%)

1.5

Predictions made of data from 6 months in time

15

10

5
Best Arm
All Arms

0

1

Number of predictions

0

2

4

6

Number of predictions

8

10

5
Best Arm
All Arms

0

10
×104

0

0.5

1

Number of predictions

Figure 6.2: Results using only user-submitted context.

34

1.5

2
×105

6.2. THE APPLICATION

6.1.3

CHAPTER 6. RESULT

Using anonymous data

The highest level of privacy consist in total anonymous data. Recommendations are
generated only through observing the flow of products, what are being bought and
when.Due to the nature of Contextual Multi-Armed Bandit Algorithms, the scenario of
anonymity can only be applied to a certain extent with the dataset used in this thesis.
One way of making anonymous recommendations can be seen as recommending the most
popular items from a certain time period. Results of this can be seen under 6.3.1.

6.2

The Application

The implemented application is implemented in a scalable manor but without any graphical interfaces, meaning that the core of the application is easily modified to use more or
less features, arms, training sets, predictions etc. Inputs are accepted following a simple
model, as can be seen in 5.4, meaning input could be taken from other systems with
some slight modifications. The application also includes support for automatic logging
and plotting of the results. Everything to make the application as easily understandable
and usable as possible.

6.3

Algorithms for Evaluation

The implemented algorithms for evaluation and comparison consist in two different approaches. The first one is through use of simple statistical models, and the second one
is a full-scale implementation of a User-Based Collaborative Filtering Algorithm.

6.3.1

Simple Statistical Models

Simple statistical models have been implemented, as described under 5.9, and used as
baselines while implementing the main application, something that have been of great
help when trying new methods early on in the implementation steps. It is however
pointless to use baselines with poor performance in the final result, which is why the
following graphs only show the baseline with the best performance. This baseline is a
statistical model based on the purchases from the last week of any point in time. It
will always recommend the ten most bought items as of the last week, to all users. The
following graphs got the same layout as the ones in 6.1, namely with a percentage of
correct recommendations on the y-axis and the total number of recommendations made
on the x-axis.

35

6.3. ALGORITHMS FOR EVALUATION

CHAPTER 6. RESULT

Predictions made of data from 1 week in time

Predictions made of data from 1 month in time
18

Correct predictions (%)

Correct predictions (%)

17
16
15
14
13
12

16

14

12

Popularity(Baseline)

11

0

1000

2000

3000

4000

5000

6000

Popularity(Baseline)

10

7000

0

0.5

Number of predictions
Predictions made of data from 3 months in time

2
×104

Correct predictions (%)

18

16

14

12

16

14

12

Popularity(Baseline)

10

1.5

Predictions made of data from 6 months in time

18

Correct predictions (%)

1

Number of predictions

0

2

4

6

Number of predictions

8

Popularity(Baseline)

10

10
×104

0

0.5

1

1.5

Number of predictions

Figure 6.3: Graphs showing recommendations based solely on popularity following the
baseline that recommends the most bought items as of the last week

36

2
×105

6.3. ALGORITHMS FOR EVALUATION

6.3.2

CHAPTER 6. RESULT

User-Based Collaborative Filtering

A full-scale implementation of an application using collaborative filtering-based algorithms have been implemented as described under 5.8. Here follow some graphs of its
performance where the graphs got the same layout as the ones in 6.1 and 6.3.1, namely
with a percentage of correct recommendations on the y-axis and the total number of
recommendations made on the x-axis.

Predictions made of data from 1 week in time

Predictions made of data from 1 month in time
6

Correct predictions (%)

Correct predictions (%)

7
6
5
4
3

5
4
3
2
1

Jaccard Similarity

2

0

1000

2000

3000

4000

5000

6000

Jaccard Similarity

0

7000

0

0.5

Number of predictions
Predictions made of data from 3 months in time

2
×104

Correct predictions (%)

6

5
4
3
2
1

5
4
3
2
1

Jaccard Similarity

0

1.5

Predictions made of data from 6 months in time

6

Correct predictions (%)

1

Number of predictions

0

2

4

6

Number of predictions

8

Jaccard Similarity

0

10
×104

0

0.5

1

1.5

Number of predictions

Figure 6.4: Graphs showing result when recommending items based on the jaccard similarity between users.

37

2
×105







