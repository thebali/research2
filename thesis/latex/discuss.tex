\chapter{Discussion}

\section{The Research Question}

In the introduction we defined the research question as:
Would the algorithmic approach of Contextual Multi-Armed Bandit Algorithms perform
well enough to belong among the different algorithmic approaches to consider when implementing recommender systems?
A justified answer to this question, with respect to the results presented in Chapter 6 and
the reasoning of the results presented in Chapter 7, is ”yes”. A reasonable conclusion that
can be made from observing the result where purchasing behaviour is predicted with a
probability of over 20\% is that considering Contextual Multi-Armed Bandit Algorithms
when implementing recommender systems is highly appropriate.
The more complex but also most suitable answer to the research question is ”probably”
or ”it depends on the setting and environment”. As the name Contextual Multi-Armed
Bandit Algorithms indicates, the algorithm makes use of contexts such as a user’s context, personal information, and might thus not be eligible in systems where all data is to
be totally anonymous or where only non-personal data exists. However, as can be seen

