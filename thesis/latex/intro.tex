\chapter{Introduction}

\section{Background}
Recommender Systems is a hot topic within Computer Science, as algorithms for the purpose of targeted advertisement get improved each year. Hence it is vital for companies to keep up with their competitors. Personal integrity, or online privacy, is also a frequently discussed topic within computer science, as well as, in society and on a global level. How much and what kind of data is ethically acceptable to observe in order to produce relevant advertisements? Is it acceptable to use all data that could be retrieved
from a user such as GPS-based movement, cookies, chat-history, previous purchases or other arbitrary personal data a company might have access to?

This thesis focuses on the evaluation and analysis of the performance of "Item-based Collaborative filtering" for the purpose of targeted advertisement within e-commerce.

\\\\Comparisons are made with more standard algorithms, being used in many live recommender systems for e-commerce as of today. Whereas the algorithmic approach of Contextual Multi-Armed Bandit algorithms have use-areas within recommender systems, very little research has been made for use within e-commerce. Performance will also be analysed and evaluated taking into consideration how much personal data are observed. How much and what kind of data can be observed and manipulated for it to still be considered socially and ethically acceptable? That is while still respecting the users’ privacy. There exist research on the subject, addressing the issue of online privacy.

\cite{park2012literature}

\section{Methodology}

\section{Related Work}

\subsection{Scope}

