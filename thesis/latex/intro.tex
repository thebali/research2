\chapter{Introduction}

\section{Background}
Recommender Systems is a hot topic within Computer Science, as algorithms for the purpose of targeted advertisement get improved each year. Hence it is vital for companies to keep up with their competitors. Personal integrity, or online privacy, is also a frequently discussed topic within computer science, as well as, in society and on a global level. How much and what kind of data is ethically acceptable to observe in order to produce relevant advertisements? Is it acceptable to use all data that could be retrieved
from a user such as GPS-based movement, cookies, chat-history, previous purchases or other arbitrary personal data a company might have access to?

This thesis focuses on the evaluation and analysis of the performance of "Item-based Collaborative filtering" for the purpose of targeted advertisement within e-commerce.

\\\\Comparisons are made with more standard algorithms, being used in many live recommender systems for e-commerce as of today. Whereas the algorithmic approach of Contextual Multi-Armed Bandit algorithms have use-areas within recommender systems, very little research has been made for use within e-commerce. Performance will also be analysed and evaluated taking into consideration how much personal data are observed. How much and what kind of data can be observed and manipulated for it to still be considered socially and ethically acceptable? That is while still respecting the users’ privacy. There exist research on the subject, addressing the issue of online privacy.

\cite{park2012literature}

\section{Methodology}

\section{Related Work}

\subsection{Scope}

Scope of the dataset explained---

The scope of the dataset intended to be used in this project consists complete database from the Amazon.com - the ecommerce website system from May 1996 - July 2014. We have decided to limit the database to the Digital Music Category only. 


The scope of the dataset intended to be used in this project consists in the complete
Junkyard database, used live in their e-commerce system from 2009 to 2014. We decided
to limit the project to using only this, as it contains enough sufficient data. The Junk-
yard dataset is very contextual. There is a huge amount of data about each user and all
of their purchases. The Junkyard dataset however has no data regarding click-sessions
from ads or products.
Moreover, the Junkyard dataset initially contains information about several million users
where the average number of orders made by each user is 3.52 and the average number
of items purchased with each order being 2.24.


