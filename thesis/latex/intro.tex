\chapter{Introduction}

\section{Background}

Recommender Systems is a hot topic within Computer Science, as algorithms for the purpose of targeted advertisement get improved each year. Hence it is vital for companies to keep up with their competitors. Personal integrity, or online privacy, is also a frequently discussed topic within computer science, as well as, in society and on a global level. How much and what kind of data is ethically acceptable to observe in order to produce relevant advertisements? Is it acceptable to use all data that could be retrieved from a user such as GPS-based movement, cookies, chat-history, previous purchases or other arbitrary personal data a company might have access to?

This thesis focuses on the evaluation and analysis of the performance of "Item-based Collaborative filtering" for the purpose of targeted advertisement within e-commerce. Standard algorithms are used for the evaluation and analysis.

---- make changes...

Comparisons are made with more standard algorithms, being used in many live recommender systems for e-commerce as of today. Whereas the algorithmic approach of Contextual Multi-Armed Bandit algorithms have use-areas within recommender systems, very little research has been made for use within e-commerce. Performance will also be analysed and evaluated taking into consideration how much personal data are observed. How much and what kind of data can be observed and manipulated for it to still be considered socially and ethically acceptable? That is while still respecting the users’ privacy. There exist research on the subject, addressing the issue of online privacy. \cite{Park2012}


\section{Methodology}
This thesis presents the methodolgy for the system in the \cite{}


\section{Related Work}

In this section, we briefly present some of the studied literature related to Multi-Armed Bandits, recommender systems, collaborative filtering, content-based filtering, demographic filtering, machine-learning, privacy and personalization.

\section{Scope}

Scope of the dataset explained---

The scope of the dataset intended to be used in this project consists complete database from the Amazon.com - the ecommerce website system from May 1996 - July 2014. We have decided to limit the database to the Digital Music Category only, as the database is sufficient for the experiment. The database also has very contextaul user data and the metadata about the product. This dataset does not contain any sort of the session based data, like click, heat zones, etc.

Amazon Review Dataset contains information about 279,899 products with 836,006 reviews from the Digital Music Category.
The average rating given by the users in the dataset is 4.54 which is quite good for the Music Industry.

\begin{verbatim}
         asin           helpful          overall    
  B004D1GZ2E:  1953   [0, 0] :550515   Min.   :1.00  
  B0026P3G12:  1926   [1, 1] : 74079   1st Qu.:4.00  
  B0000AGWEC:  1823   [2, 2] : 26928   Median :5.00  
  B004K4AUZW:  1527   [0, 1] : 25300   Mean   :4.54  
  B000BGR18W:  1386   [1, 2] : 16855   3rd Qu.:5.00  
  B00008H2LB:  1325   [3, 3] : 13226   Max.   :5.00  
\end{verbatim}


