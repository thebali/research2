\chapter{Methodology}

\section{Dataset}

\subsection{Data Gathering}
In order to get the required material to achieve this dissertation, files containing consumer reviews for diverse products derived from Amazon.com were used. These files were collected by Julian McAuley, researcher at the University of California, and were available on the following website: http://jmcauley.ucsd.edu/data/amazon/. The extracted files only represented a subset of the data in which all items and users had at least 5 reviews (5-core). Duplicates, accounting for less than 1 percent of reviews, were removed.

This dataset contains product reviews and metadata from Amazon, including 142.8 million reviews spanning May 1996 - July 2014.

This dataset includes reviews (ratings, text, helpfulness votes), product metadata (descriptions, category information, price, brand, and image features), and links (also viewed/also bought graphs). 

\section{Implementation}

SVD()


• Always recommend items based on gender. (Men and women always buy the same
items as other men and women, respectively)
• Always recommend items based on age. (People at a certain age always buy same
items as other people in the same age)
• Always recommend items based on domicile. (People from city X always buy the
same items as other people from city X)
Baselines have been used before any other extensive evaluation method, something that
has been of great help when implementing the application, as these baselines can be seen
as evaluation in its most basic form.

