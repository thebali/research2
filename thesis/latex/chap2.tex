\chapter{Recommender System}
Recommender system is and remains a hot topic within computer science in 2015. Huge conferences are hosted in different parts of the world annually, featuring many big companies. The RecSys conference [33] is a good example
of this. Recommender Systems has a wide use-area within all kinds of advertisement on the web when specifically targeted ads are needed. These ads can be seen everywhere from websites and apps to emails, text-messages etc. In the list of big actors on the stage of recommender systems, hardly surprisingly, one find companies such as Facebook, Google etc. But Recommender Systems are also used for recommending music in for instance Spotify and movies/TV-shows in Netflix. There are a variety of different algorithms for each use-area. The setting of which this thesis will focus consists
in algorithms for recommender systems within garment-based e-commerce. Here follows some of the more common general techniques used in recommender systems.


\section{Collaborative Filtering}
The basic idea of collaborative filtering is to find information or patterns using collaborative techniques among multiple data sources. Data sources typically consist of users and items such as movies, songs etc. Within e-commerce, and this project specifically, those data sources mainly consist of users, and products that can be bought by these users. Finding these patterns is accomplished through collecting and analysing huge amounts of data on users’ behaviours and activities as well as items. Within e-commerce those activities and behaviours include but are not limited to purchases and click-able ad- vertisements. Collaborative filtering[18] is one of the most common general methods for recommender systems being applicable within several big use-areas. Collaborative Filtering is frequently used in social networks such as Facebook, LinkedIn, mySpace, Twitter etc to effectively recommend new friends, pages, groups as well as who to follow or what to like. But also applications such as Youtube, Reddit, Netflix etc make use of collaborative filtering. Collaborative filtering is viable in all applications where one can observe connections between a user and their registered friends or followers. But it is also widely used within e-commerce, where Amazon is a good example who popularised algorithms for item-to-item based collaborative filtering [21].

