\chapter{Recommender System}
Recommender system is and remains a hot topic within computer science in 2015. Huge conferences are hosted in different parts of the world annually, featuring many big companies. The RecSys conference [33] is a good example
of this. Recommender Systems has a wide use-area within all kinds of advertisement on the web when specifically targeted ads are needed. These ads can be seen everywhere from websites and apps to emails, text-messages etc. In the list of big actors on the stage of recommender systems, hardly surprisingly, one find companies such as Facebook, Google etc. But Recommender Systems are also used for recommending music in for instance Spotify and movies/TV-shows in Netflix. There are a variety of different algorithms for each use-area. The setting of which this thesis will focus consists
in algorithms for recommender systems within garment-based e-commerce. Here follows some of the more common general techniques used in recommender systems.


\section{Collaborative Filtering}
The basic idea of collaborative filtering is to find information or patterns using collaborative techniques among multiple data sources. Data sources typically consist of users and items such as movies, songs etc. Within e-commerce, and this project specifically, those data sources mainly consist of users, and products that can be bought by these users. Finding these patterns is accomplished through collecting and analysing huge amounts of data on users’ behaviours and activities as well as items. Within e-commerce those activities and behaviours include but are not limited to purchases and click-able ad- vertisements. Collaborative filtering[18] is one of the most common general methods for recommender systems being applicable within several big use-areas. Collaborative Filtering is frequently used in social networks such as Facebook, LinkedIn, mySpace, Twitter etc to effectively recommend new friends, pages, groups as well as who to follow or what to like. But also applications such as Youtube, Reddit, Netflix etc make use of collaborative filtering. Collaborative filtering is viable in all applications where one can observe connections between a user and their registered friends or followers. But it is also widely used within e-commerce, where Amazon is a good example who popularised algorithms for item-to-item based collaborative filtering [21].

Most Collaborative Filtering techniques can be expressed by the two general steps, Similarity Computation and Prediction Generation, described under 2.2

\subsection{User based collaborative filtering}
The main idea of recommender systems built on user-based collaborative filtering consists in computing the similarity between users’ {u,j} profiles, s u\textsubscript{j} . That is, computing a prediction for the probability of the user u liking a specific item i, consists in computing a rating based on all ratings made by users with similar profiles. All similar profiles contribute to this prediction depending on the similarity factor, s uj [34, 35, 36]. 

Typically this can be reduced to the two general steps. 
1. While observing a user’s actions in a system, look for similar users with equal behaviour- and activity-patterns. 
2. Use the observations from step 1 to compute a prediction for the specified user. 

See 2.2 for more information regarding Similarity Computation and Prediction Genera- tion techniques.

\subsection{Item based collaborative filtering}
The concept of item-based collaborative filtering applies the same idea as its User-based counterpart, but the similarity is computed between items instead of users, and is usually described as "Users who bought this also bought that". 

More generally, taking an item-based approach means looking into the set, I, of items a specific user, u, has rated, using their context to compute the similarity of other items, \{i\textsubscript{0} ,i\textsubscript{1} ,...,i\textsubscript{n} \}, not in I. Their corresponding similarities are computed at the same time as {s i0 ,s i1 ,...,s in }. When these similarities and their corresponding items have been found, a prediction can be computed. [36, 37, 38]. 
This can typically be split into three steps: 
1. Construct a User-Item matrix, M[u][i], giving each index of the matrix a rating, r ui , on an item i performed by a user u. 
2. Compute the similarity, s i\textsubscript{1}, i\textsubscript{2}, of two items i 1 and i 2 by looking at co-rated pairs from different users.
3. Generate predictions based on some prediction method, described under 2.2. Item-based collaborative filtering is a very common approach, often used together with algorithms for matrix factorization.

\subsection{Matrix Factorization}
Matrix factorization is an algebraic operation consisting in factorising a matrix M, meaning finding the matrices M 1 ,M 2 ,...,M 3 such that when they are multiplied, the resulting matrix is M. In collaborative filtering based recommender systems this can be used as a rather simple and very intuitive algorithm for discovering latent features. By constructing a User-Item matrix M, where each index contains a rating r on an item i performed by a user u:



\begin{equation*}

\textbf{M\textsubscript{u,i}} = $\left(
\begin{array}{cccc}

{r_u__1i_1 & {r_u}__1i_2 & \ldots & r_u__1i_n \\
r_u__2i_1 & r_u__2i_2 & \ldots & r_u__2i_n \\
\vdots & \vdots & \ddots & \vdots \\
r_u__mi_1 & r_u__mi_2 & \ldots & r_u__mi_n

\end{array}
\right)$

\end{equation*}

\paragraph
this is the equation I am refering to ...\eqref{ratingmatrix}

