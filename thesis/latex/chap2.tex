\chapter{Recommender System}
Recommender system is and remains a hot topic within computer science in 2015. Huge conferences are hosted in different parts of the world annually, featuring many big companies. The RecSys conference [33] is a good example
of this. Recommender Systems has a wide use-area within all kinds of advertisement on the web when specifically targeted ads are needed. These ads can be seen everywhere from websites and apps to emails, text-messages etc. In the list of big actors on the stage of recommender systems, hardly surprisingly, one find companies such as Facebook, Google etc. But Recommender Systems are also used for recommending music in for instance Spotify and movies/TV-shows in Netflix. There are a variety of different algorithms for each use-area. The setting of which this thesis will focus consists
in algorithms for recommender systems within garment-based e-commerce. Here follows some of the more common general techniques used in recommender systems.


\section{Collaborative Filtering}
The basic idea of collaborative filtering is to find information or patterns using collaborative techniques among multiple data sources. Data sources typically consist of users and items such as movies, songs etc. Within e-commerce, and this project specifically, those data sources mainly consist of users, and products that can be bought by these users. Finding these patterns is accomplished through collecting and analysing huge amounts of data on users’ behaviours and activities as well as items. Within e-commerce those activities and behaviours include but are not limited to purchases and click-able ad- vertisements. Collaborative filtering[18] is one of the most common general methods for recommender systems being applicable within several big use-areas. Collaborative Filtering is frequently used in social networks such as Facebook, LinkedIn, mySpace, Twitter etc to effectively recommend new friends, pages, groups as well as who to follow or what to like. But also applications such as Youtube, Reddit, Netflix etc make use of collaborative filtering. Collaborative filtering is viable in all applications where one can observe connections between a user and their registered friends or followers. But it is also widely used within e-commerce, where Amazon is a good example who popularised algorithms for item-to-item based collaborative filtering [21].

Most Collaborative Filtering techniques can be expressed by the two general steps, Similarity Computation and Prediction Generation, described under 2.2

\subsection{User based collaborative filtering}
The main idea of recommender systems built on user-based collaborative filtering consists in computing the similarity between users’ {u,j} profiles, s u\textsubscript{j} . That is, computing a prediction for the probability of the user u liking a specific item i, consists in computing a rating based on all ratings made by users with similar profiles. All similar profiles contribute to this prediction depending on the similarity factor, s uj [34, 35, 36]. 

Typically this can be reduced to the two general steps. 
1. While observing a user’s actions in a system, look for similar users with equal behaviour- and activity-patterns. 
2. Use the observations from step 1 to compute a prediction for the specified user. 

See 2.2 for more information regarding Similarity Computation and Prediction Genera- tion techniques.

\subsection{Item based collaborative filtering}
The concept of item-based collaborative filtering applies the same idea as its User-based counterpart, but the similarity is computed between items instead of users, and is usually described as "Users who bought this also bought that". 

More generally, taking an item-based approach means looking into the set, I, of items a specific user, u, has rated, using their context to compute the similarity of other items, \{i\textsubscript{0} ,i\textsubscript{1} ,...,i\textsubscript{n} \}, not in I. Their corresponding similarities are computed at the same time as {s i0 ,s i1 ,...,s in }. When these similarities and their corresponding items have been found, a prediction can be computed. [36, 37, 38]. 
This can typically be split into three steps: 
1. Construct a User-Item matrix, M[u][i], giving each index of the matrix a rating, r ui , on an item i performed by a user u. 
2. Compute the similarity, s i\textsubscript{1}, i\textsubscript{2}, of two items i 1 and i 2 by looking at co-rated pairs from different users.
3. Generate predictions based on some prediction method, described under 2.2. Item-based collaborative filtering is a very common approach, often used together with algorithms for matrix factorization.

\subsection{Matrix Factorization}
Matrix factorization is an algebraic operation consisting in factorising a matrix M, meaning finding the matrices M 1 ,M 2 ,...,M 3 such that when they are multiplied, the resulting matrix is M. In collaborative filtering based recommender systems this can be used as a rather simple and very intuitive algorithm for discovering latent features. By constructing a User-Item matrix M, where each index contains a rating r on an item i performed by a user u:

\[  
\label{eq:ratingmatrix}
\textbf{M\textsubscript{u,i}} = \left(
\begin{array}{cccc}
r_{{u_1}{i_1}}&r_{{u_1}{i_2}}& \ldots&r_{{u_1}{i_n}} \\
r_{u_2}{i_1}&r_{{u_2}{i_2}}& \ldots&r_{{u_2}{i_n}} \\
\vdots & \vdots & \ddots & \vdots \\
r_{{u_m}{i_1}}&r_{{u_m}{i_2}}& \ldots&r_{{u_m}{i_n}}
\end{array}
\right)
\]

For example, by limiting the matrix to 4×4 and adding some fictitious values for ratings
where - symbolises yet unrated items, a matrix to be factorised might look something
like \eqref{ratingmatrix}


\[  
\label{eq:numbermatrix}
\textbf{M\textsubscript{u,i}} = \left(
\begin{array}{cccc}
5&1&\-&4 \\
3&\-&\-&3 \\
\vdots & \vdots & \ddots & \vdots \\
2&\-& \ldots&4
\end{array}
\right)
\]

Matrix factorisation can be seen as the task of predicting the missing ratings meaning filling in the blanks (-) in matrix 2.2. Latent features of items liked by different users might be different types of cloths or accessories which might be a shared interest between some users. So for the matrix, 2.2 above, an assumption regarding how many latent features one wishes to find must be made. Let us assume we want to find k latent features and we have the sets U and I containing all users and all items respectively.

This means finding two matrices, N(\textit{a  |U| × k matrix }) and O(\textit{an  |I| × k  matrix}) which when multiplied approximating M:

\[ M \approx N \times O = \hat{M} \]

Further, using similarity and prediction computing techniques as described under 2.2, the gaps can be filled in and thereby resulting in really good recommendations [39].

\section{Similarity Calculation and Prediction Generation}
Similarity and prediction computations compose vital parts of collaborative filtering based recommender systems. Something that can be seen in 2.1.1, 2.1.2 and 2.1.3. The similarity computation is performed before the prediction generation and the basic idea is to first isolate users who have rated two different items, and secondly to apply some similarity computation technique between these two items to determine their similarity. There are several ways of computing the similarity but some common methods are Cosine-based Similarity, Correlation-based Similarity and Adjusted Cosine-based Similarity, Jaccard Similarity, Jaro-Winkler Similarity, Sörensen Similarity.
When the similarity computation is completed, it is time for the most important part in a collaborative filtering based recommender system; generating the output in terms of prediction. As for computing the similarity there are a number of techniques to choose from when generating predictions. Some of the more common methods are using regression or the weighted average/weighted sum.

\section{Content Based Filtering}\label{contentfiltering}
Another well known method when implementing recommender systems is content-based filtering[16, 17]. Content-based filtering is commonly used for movie recommendations or within other environments where for instance keywords are used to describe the items of the system. Among the actors of recommendation systems using content-based filtering one can find The Internet Movie Database and other popular websites for movies. The methods of Content-based filtering commonly make use of the correlation between item features and preferences from a user’s profile, as opposed to the collaborative filtering approach that selects items based on the correlation between users with similar profiles. 

For the above to work, the system needs to deploy some learning technique [16, 17] such as Bayesian networks, clustering, decision trees, neural networks, reinforcement learning, Nearest Neighbour etc. The system uses these techniques when observing historical data of users to learn their preferences. The intention is that, after sufficient amounts of data have been observed, the system should be able to predict future behaviour of a specific user.


\subsection{Selecting a (Learning) Model}\label{trainingmodel}
A key component when implementing a recommender system using content-based filtering is choosing a method for how the algorithm should be trained [16, 17], using available data from the system. Using this data to create a model of a user’s preferences and history in the system can be seen as a kind of classification problem. There are several significantly different methods suitable for solving this, some of which are mentioned in \autoref{contentfiltering}.

Which one to choose depend on the setting for which it is going to be used. For instance, Bayesian networks are commonly practical in a setting where knowledge about users change slowly, relative to time needed to construct the model. Clustering techniques on the other hand has a tendency to generate less-personal predictions than other methods[40]. Although, due to the nature of clustering, once the clustering process is complete, the new groups of data to be analyzed are significantly smaller than before and performance, in terms of computation time, are thereby likely to be good. Decision trees have shown a tendency of basing classifications on as few instances as possible. Something that has lead to worse performance in the sense of accuracy[41]. Although with a smaller number of structured attributes, the performance and simplicity of decision trees are all advantages when applied in content-based recommender systems. Nearest Neighbour methods are commonly known as old and reliable go-to methods when nothing else works. It has the worst performance in most cases, but are most likely to succeed[17].

\section{Demographic Filtering}
Recommender systems built around the concept of demographic filtering work very much like their content-based counterpart, although only making use of personal data provided by the users themselves through a registering process, survey response, purchase history etc. This rather than observing and learning user behaviour to classify the users depending their purchase history, ratings etc[20, 42]. A Demographic filtering approach does not, in contrary to content-based filtering, apply a user preference model. It does however apply an item preference model for the items of a system. This basically means that a demographic approach is more privacy-preserving than a content-based filtering one[19].

\section{Other recommendations techniques}
The techniques described under 2.1, 2.3 and 2.4 are the most common when implementing recommender systems [20, 43]. More approaches have been studied and considered, but none that are to be considered more common than the ones described in 2.1, 2.3 and 2.4. One example is using techniques of knowledge-based filtering described in[44].

\section{Hybrid Recommender Systems}
A hybrid approach to implementing recommender system means designing the system so as to making use of several recommendation techniques such as for instance collaborative filtering techniques and content based filtering, making predictions from the combined conclusions of the two. This can be done by unifying the two techniques, by adding features from one into the other or simply by running algorithms for both techniques separately and then combine the results in some way. The most common example of a hybrid based recommender system is the one used by Netflix. While an environment like Netflix is well suited for a hybrid recommender system, it does not fit everywhere.

Why Netflix is considered a hybrid system:

\begin{itemize}
	\item {\textit{Collaborative Filtering}: Observing the watching and browsing habits of similar users.}
\item \blindtext
\item {\textit{Content-Based Filtering}: Observing users with equal preferences and how they rated certain movies.}

\end{itemize}

\section{Implementation Challenges in Recommender Systems}

\subsection{Grey/Black Sheep}
These terms ”grey sheeps”, ”black sheeps” are used to refer to the users in the recommender systems. Grey sheep refer to user whose characteristics do not overlap with any other user or group of users, that is not consistently agreeing or disagreeing with any other group of users. Black sheep on the other hand refer to users with extremely varying taste patterns. Something that make predictions nearly impossible.

\subsection{Sparsity and Subjective Problems}
The sparsity problem applies in collaborative filtering based system when it is hard to
find items rated by enough people to be feasible to consider (The item-user Matrix is
very large and sparse). This is common as the amount of items in most recommender
environments exceeds the amount a user is able to explore by far. However sparsity
problems in different forms might also be encountered in systems based on content-based
filtering. Algorithms for that technique cannot see information as objective or subjective,
meaning it cannot distinguish between for instance irony and actual opinions. Which in
turn means that a learning algorithm might draw faulty conclusions.

11

2.7. CHALLENGES

2.7.3

CHAPTER 2. RECOMMENDER SYSTEMS

Shilling attacks

In recommender systems where actual ratings constitutes the core of predictions, it is
often necessary to introduce precautions to discourage manipulation attempts. These
kinds of manipulations might occur in systems where everyone can make ratings, whereas
biased users might give lots of positive feedback for products related to themselves and
unjustified negative feedback for their competitors.

2.7.4

Cold Start

In a system based on collaborative filtering, new items with no ratings will most likely
not be predicted for anyone until it has been rated by several users. The same applies
to new users of a specific system. If users do not have any recorded activities on which
to base predictions, most recommendations will most likely be inaccurate. Likewise a
content-based filtering system will have issues suggesting accurate recommendations of
items that the behaviour of a user do not provide evidence for. Additional techniques
need to be added to give the recommender system capabilities for solving this.

2.7.5

Lack of computational power

A common challenge when implementing algorithms for collaborative filtering-based recommender systems is the lack of computational power on a single machine whereas most
will suffer from scalability problems. Something that is easily understood by imagining that there are systems containing millions of items and even more users, meaning
that even algorithms with polynomial or even linear time complexity will be slow. For
instance in online environments where recommendations need to be done in real time,
without any offline computations, and 2-3 seconds are already too slow.

2.7.6

Equal items with different names

In many systems with millions of items it is pretty common that the same or similar items
exist several times in the database, but still have different names or even same names
but different ids. Unless thought about, this could be a problem in many recommender
systems as they would be treated as different items.

2.7.7

The Exploration VS Exploitation dilemma

A returning issue when working with implementing recommender systems is the exploration versus exploitation trade-off which can be seen mostly in machine-learning
environments and is something that every developer needs to consider [46]. Exploration
is the task of acquiring new knowledge about an environment or setting while exploitation means using existing knowledge to make decisions or predictions. The dilemma
consists in balancing these tasks over time i.e. is it worth risking more exploration for a
better reward rather than exploiting already sufficient knowledge about the environment
and thereby already getting a good reward with some probability.
12

2.7. CHALLENGES

CHAPTER 2. RECOMMENDER SYSTEMS

And what does this mean when implementing recommender systems for use within ecommerce? An example would be if, let us for simplicity, say that an algorithm used by a
recommender system has found one or perhaps a few products that get sold with a high
probability every time they are recommended to some type of users. This probability
is likely to decrease over time, but how does the algorithm know when to stop recommending the products with known decent profit in favour of exploring new products?

13

3
Multi-Armed Bandit Algorithms
he multi-armed bandit problem describes how a gambler is standing in
front of multiple slot machines (one-armed bandits), needing to make decisions regarding which arm to pull with the intention of maximising the profit.
In the sense of recommender systems, this can be modelled as an actor or
algorithm having multiple models to choose from, whereas the algorithm needs to decide which model to choose in order to give the best recommendation or make the best
prediction.

T

3.1

Background

The Multi-Armed Bandit problem is a decision-making problem for deciding on a model
to use when making predictions about the future[7]. It originates from the Markov decision process and was first formulated in 1952[8], however its concepts have documented
discussions from earlier and it at this point in time was not spoken of as Multi-Armed
Bandits. Since then there have been numerous different algorithms for how to solve it
with good performance.
Some of the bandit models’ practical use-areas have, for instance, consisted in providing
predictions regarding which projects that are most likely to be successful within different
research areas. This helps when determining how investments should be made to maximise the profit. It can also provide useful information regarding failing projects, so as
to stop the flow of money in an early stage. Other practical applications for the bandit
model have been clinical trials and adaptive routing, where it was used for optimisation.

3.2

The Algorithm

A general idea of The Multi-Armed Bandit problem is as follows:
14

3.2. THE ALGORITHM

CHAPTER 3. THE MAB ALGORITHM

1. Each arm is to be modelled as a lever, which upon being pulled, provides a reward
r ∈ R independently, from its own distribution in a setting where all arms have
their own distributions.
2. Mean values µ1 , µ2 , . . . , µn can then be computed, where n is the cardinality of the
set of distributions for all arms.
3. An agent plays on one lever at a time iteratively and observes the associated
rewards for each arm, respectively.
4. The objective of the agent is to maximise her winnings, the sum of the collected
rewards.
Within decision theory and probabilistic modelling, regret bounds are often used for
measuring negative emotion experience. That is when learning how a different course
of action would have resulted in a more favourable outcome. It is thereby desirable to
implement algorithms with as low regret bound as possible.
By modelling the Multi-Armed Bandit problem as a one-state Markov decision chain,
the total regret ρ after m rounds can be seen as the expected difference between the sum
of total rewards from an optimal solution and the sum of the actual collected rewards
ρ = mµ∗ −

m
X

rbi

(3.1)

i=1

where rbi is the reward of the ith iteration.
Algorithm 1: Multi-armed bandit algorithm
Data: A: the arms, P: the purchases
1 previousP urchases ← {}
2 forall the p in P do
3
a ← policyLogic(A, previousP urchases) // Choose an arm depending on
policy
4
ListOfItems ← pullArm(a)
5
if p contains ListOfItems then
6
reward ← 1
7
else
8
reward ← 0
9
end
10
updateArm(a, reward) // Update the arm depending on success or failure
11
append(previousP urchases, p) // Append purchase p to the previous
12 end
policyLogic returns an arm a based on the policy logic described in section 3.3. pullArm
returns a list of items that a wants to recommend. updateArm is then updating a depending on whether the recommendations were successful or not. A recommendation is
15

3.3. POLICIES

CHAPTER 3. THE MAB ALGORITHM

considered successful if a recommended item is purchased by the specific user at a later
point in time. The update function is implemented differently for different implementations of the algorithm but the most common way is having α and β variables that
get updated with success or failure. These parameters are then taken into consideration
when policyLogic is executed.

3.3

Policies

The main problem, using Multi-Armed Bandit algorithms, is to maintain a good ratio
between exploration versus exploitation, something that has been proved necessary to
consider in order to get as high cumulative profit as possible.

3.3.1

-greedy

The -greedy[47] policy works by having a fixed value, , that determines how much the
algorithm should explore and/or exploit. The policy starts with generating a random
number x. If the randomly generated value x is below  it chooses an arm at random,
else it takes the arm with the best performance so far.
Algorithm 2: Multi-armed bandit algorithm with the -greedy policy
Data: A: the arms, P: the purchases, βa : tries of arm a, αa : successful tries of
arm a
1 Initialise all αa and βa to 0
2 forall the p in P do
3
x ← random[0,1]
4
if x <  then
5
a ← random(A)
6
else
7
a ← argmax( αβaa )
8
end
9
ListOfItems ← pullArm(a)
10
if p contains ListOfItems then
11
inc(αa )
12
else
13
inc(βa )
14
end
15 end

3.3.2

Upper Confidence Bound

Upper confidence bound does not, in contrary to the -greedy policy, continuously explore
throughout the whole running time of the algorithm with a fixed probability. Instead it
has a discovery phase, where it tries each arm once. After the discovery phase is finished,

16

3.3. POLICIES

CHAPTER 3. THE MAB ALGORITHM

it generates the policy for choosing an arm as
µa + λσa

(3.2)

where µa is the estimated reward for arm a, and σa a confidence bound measuring the
accuracy of our estimate, mean µa . The λ is, just as when using the -greedy policy, a
fixed parameter to balance exploration versus exploitation.
A very simple and easy implementation, UCB1[48], can be performed by setting µa = αβaa
q P
and λσa = 2·lnβa β where αa is the number of times a specific arm has succeeded in
predicting a future purchase of an item, and βa is the number of times the arm has tried
to predict items in total.
Algorithm 3: Multi-armed bandit algorithm with the UCB1 policy
Data: A: the arms, P: the purchases, βa : tries of arm a, αa : successful tries of
arm a
1 Initialise all αa and βa to 0
2 forall the p in P do
q
a ← argmax( αβaa +

3
4
5
6
7
8
9
10

P
2·ln β
)
βa

ListOfItems ← pullArm(a)
if p contains ListOfItems then
inc(αa )
else
inc(βa )
end
end

3.3.3

Thompson Sampling

A policy on which a lot of recent focus have been put is a policy known as Thompson
Sampling [25, 26, 49, 50, 51], which unlike the UCB1 policy goes back to the exploration
phase faster when a previously well performing arm starts failing. It is also different in
comparison with the so called -greedy policy which constantly keeps on exploring by
some fixed factor. Using Thomson Sampling exploration is instead decaying over time,
as the algorithm finds one or more arms that outperforms the others. If, however, there
is a change in the market, and a previously good arm suddenly starts to fail more often,
the method of Thompson Sampling will once again take a more exploratory approach as

17

3.4. CONTEXTUAL MAB

CHAPTER 3. THE MAB ALGORITHM

explained above.
Algorithm 4: Multi-armed bandit algorithm with Thompson Sampling without
decay
Data: A: the arms, P: the purchases, βa : tries of arm a, αa : successful tries of
arm a, B: beta-distribution function, λ: decay-factor
1 Initialise all αa and βa to 1
2 λ = 1.0
3 forall the p in P do
4
forall the a in A do
5
sa ← B(αa , βa )
6
end
7
a ← arg max (sa )
8
ListOfItems ← pullArm(a)
9
βa ← βa · λ
10
αa ← αa · λ
11
if p contains ListOfItems then
12
inc(αa )
13
else
14
inc(βa )
15
end
16 end
The goal of the Multi-Armed Bandit algorithm is to determine on which model to rely,
that would work best for the system at a specific point in time. Having the algorithm
choose among different statistical models, depending on what seems to be best at this
now, takes care of the problem where a specific method or formula which seemed to
work great one month might fail horribly the month after. This because of the algorithm
always choosing the best out of all available models, depending on actual premises from
this specific point in time. To be able to achieve this setting, the system needs to be selfupdating, something that Thompson Sampling is by default. The α and β parameters in
the algorithm can easily be adjusted to decay over time, simply by setting a decay-factor
λ to less than 1.0. If the intention is to trust old results rather than new, λ can also be
set to values higher than 1.0.

3.4

The Contextual Multi-Armed Bandit

The Contextual Multi-Armed Bandit Algorithm is a slight modified version of the original problem [11, 15]. Considering the phased solution in 3.2, for the Contextual MultiArmed Bandit, another phase is added between step two and step three. In addition to
the normal setting of a Multi-Armed Bandit there is now additional information that
might influence the choice of an agent regarding which arm to pull. Before actually
deciding, the agent now sees an n-dimensional feature vector, from now on referred to
as context vector, for each arm. The agent then uses these context vectors together

18

3.4. CONTEXTUAL MAB

CHAPTER 3. THE MAB ALGORITHM

with the rewards of each arm respectively upon deciding on what arm to pull in the
current iteration. The agent’s goal is to, overtime, gather enough information about the
relation between context vectors and rewards, so as to being able to predict which arm is
gonna yield the biggest reward only through observing the features of the context vectors.
The users, items and arms all have unique context vectors. When an item is bought
by a user, the context vector is updated for both the user and the item. If the arm
that was assigned the task of recommending an item made a successful prediction, the
system will update its α, as described in 1, to remember that the current arm had a
good context vector and should be used more often.

19

4
Privacy-Preserving Aspects
ersonal integrity is a frequently discussed topic in the connected world.
Most new products today come with the feature of being able to use the Internet; phones, clocks, cars, TVs, baby monitors and even entire households etc.
There are more devices connected to the Internet than there are people living
on the planet earth [52]. Being able to stay connected around the clock every day of
the month has shed some light on new ethical dilemmas and issues. Internet started
out as a place where its users were more or less anonymous. But despite the efforts in
present time, creating techniques so as to be more or less anonymous online, a very small
portion of the connected world actually make use of these kind of techniques. Applications on our smart devices ask for much more privileges than what their functionality
requires, most of the time. This, however, is the case for most online-based systems in
general. People are naive most of the time so as to trust systems and applications with
whatever personal information they might ask for. This kind of data can then be sold
to companies, authorities or in the worst case, hacked or stolen.

P

4.1

General Aspects of Personal Integrity

Smart phone applications are a big part of many people’s lives. They are used for browsing social medias, calendar and scheduling, public transport, work-out, calorie-tracking,
storage, e-commerce and just about anything else. Commonly applications do not ask
for any privileges when they are new. However, as the amount of users increase one
can be certain that sooner or later the application might ask for extended privileges [53]
such as access to files, text-messages, contacts and in some cases even camera and GPS.
Most people do not see an issue with this, if it is a good application. In many cases this
might not even be a problem. Commonly, regular people, would not complain about an
applications performing movie, TV-shows or shopping recommendations despite using
all possible private data, as long as the recommendations are valid. In many cases, data

20

4.2. USER-APPROACH

CHAPTER 4. PRIVACY-PRESERVING ASPECTS

changes hands every now and then. Mostly probably because of being sold. What most
people do not think of is the fact that once data hits the Internet, it is out of control.
Most of the time one can never be sure where its personal data ends up. The more
places containing our personal data, the bigger the chance is of it getting leaked, hacked
or stolen[54].
These issues are obviously not limited to smart phone applications. It concerns other
applications that possess personal information as well. It could be games, e-commerce
systems, normal web sites and systems connected to the web in general. It becomes
relevant in e-commerce platforms that make use of complex recommender systems and
personalised predictions, as those personalised predictions are based on personal information, actions and behaviour in the system. See Chapter 2 for more about recommender
systems.
To this comes ethical dilemmas; People actively choose to provide companies with their
personal data through consistent use of their applications and products. From the aspect
of the groups possessing this kind of data, what is ethically acceptable to do with it?
Could it be sold or distributed freely? See [55] and [56] for further research on this topic.

4.2

A Privacy-Preserving User Approach

There are several techniques, of which most are very easy to use, the regular user can
apply to achieve good privacy when browsing the web. Other than common sense such
as to be careful and critical, normal methods are through use of proxy servers and virtual
private networks. One of the more popular techniques as of last years are through use
of onion routing protocols [57].
Onion routing means using layers of encryption where each layer consists of decryption at individual nodes. After decryption is done at each individual node the next
destination node, in the chain of nodes, is uncovered. As the message reaches the final
destination there should be only one layer of encryption left, which when decrypted provides the original message. Because of each individual node only knowing the location of
the immediately preceding and following nodes, the original sender remains anonymous.
To create and transmit the message initially, an initiator node selects a set of nodes
from a list provided by a leader node. The initiator node then obtains a public key
from the leader node to send an encrypted message to the first node in the chain, using
asymmetric key cryptography to establish a connection and a shared session key[58].

4.3

Privacy-Preserving Systems

Privacy-Preserving systems is an area where much research has been made the last
century[5, 19, 59], but despite this very little of the research is being applied in commercial systems. And this for several reasons such as companies wanting to make use of

21

4.3. SYSTEMS

CHAPTER 4. PRIVACY-PRESERVING ASPECTS

personal features to improve their systems and the user-experience as a whole, but also
for more controversial reasons as those mentioned in 4.1.

4.3.1

Developing a Privacy-Preserving System

Generally you could describe a privacy-preserving system as a system which is developed
so as to actively avoid using or even asking for personal data[19, 59]. Although, privacypreserving systems can contain various levels of privacy where the highest level of privacy
would be total anonymity. Obviously some systems are more privacy-preserving than
others. Examples of lower levels of privacy are systems that only make use of usersubmitted data or data that a user actively choose to provide the service or system
with. Among the lowest levels of privacy are systems that record user-activity and behaviour of actions performed by users. Personal data that can be used, perhaps by
the system itself, to classify users in ways that might or might not be more or less
ethically acceptable.

22

5
Implementation
he implementation consists in a full scale implementation of the Contextual
Multi-armed bandit algorithm using Thompson Sampling, data extraction,
construction of context vectors as well as algorithms for evaluation. Because of
each implementation of a Multi-Armed Bandit algorithm being unique depending on the actual use-area and dataset, this chapter will focus on an implementation for ecommerce, and more specifically the Junkyard dataset. The implemented recommender
system is built exclusively with Multi-Armed Bandits with its associated techniques
having a user experience-based approach rather than a money-making one. The repository for the full code-base can be found at https://github.com/FredrikEk/MasterThesis-2015-Contextual-Multi-Armed-Bandits.

T
5.1

Approach

As can be seen in chapter 2 the diversity of recommender systems with methods and
algorithms for implementing them are huge and rather complex. The implementation
performed in this thesis takes a user-centric approach taking most influences from a
content-based filtering approach. Due to the algorithmic and machine-learning nature
of bandit algorithms, thinking in the ways of content-based filtering when applying
methods of Contextual Multi-Armed bandit algorithms comes naturally.

5.2

Thompson Sampling

The implementation of Multi Armed Bandits in this thesis uses Thompson Sampling
as the arm learning-policy. It helps the algorithm to update which arm being the best
at a specific point in time on-the-fly while being able to predict different recommendations even without any offline calculations. In contrary to the UCB method, Thompson
Sampling generate items randomly, which means that it could have an advantage in
23

5.3. ALGORITHM

CHAPTER 5. IMPLEMENTATION

e-commerce systems where it would be bad if the same set of items was returned repeatedly until a recalculation is made. It is however an open question, whether this is the
case or not, that needs to be evaluated with fixed static data, as is the case for this thesis. Something that could be done using synthetic data modelling (7.2.1), but that have
not been done in this thesis. The UCB method requires an offline recalculation before
new items are recommended. Why this is bad is extra obvious in a setting where a web
page might get updated every time a user navigates to a new view, as using Thompson
Sampling, would mean generating different items each time a page is updated/reloaded
and the algorithm thereby has a higher chance of predicting something that the user
finds interesting and perhaps even wants to buy.

5.3

Algorithm

For each purchase (session) the algorithm initially draws a random sample θ from a beta
distribution, using each arm’s α- and β- variables. The arm arm with highest θ will
be the arm used to perform the prediction for this session. In order to find the top
ten items, considered the best by arm, the algorithm uses the contextual vector of the
current user user together with all the items in the dataset. The contextual vector of
user is element-wise multiplied with the contextual vector of an item item, which in
turn is scalar multiplied with the contextual vector of arm. The resulting value is a
measurement of how good item is for user according to arm. The ten items with the
highest measurement values then gets recommended. If any of these items are bought
by user, the prediction of arm is considered correct and the α parameter of arm is
increased by one. On the other hand if the recommended items turn out to be faulty

24

5.4. DATA EXTRACTION

CHAPTER 5. IMPLEMENTATION

predictions the β is increased by one, whereas the α is left unchanged.
Algorithm 5: Multi-armed bandit algorithm
Data: A: the arms, P: the purchases, I: all the items, U: all the users
1 forall the p in P do
2
forall the a in A do
3
θa ← BetaDistribution(αa , βa )
4
end
5
arm ← arg max (θa )
6
user ← puser
7
forall the item in I do
8
itemreward = armcontext · (usercontext ∗ itemcontext )
9
end
10
ListOfItems ← top10 arg max (itemreward )
11
itemsBought ← pitems
12
if itemsBought contains ListOfItems then
13
reward ← 1
14
else
15
reward ← 0
16
end
17
updateArm(arm, reward) // Update the arm depending on success or failure
18
updateItems(itemsBought, user) // Update the items which were bought
19
updateU ser(user, itemsBought) // Update the user who bought items
20 end
For more information on how the context vectors, armcontext , usercontext , itemcontext )
work and are built up see section 5.5.

5.3.1

Choosing the number of arms

In the implemented application of this thesis, the algorithm use 50 arms. A good unofficial measurement that has been found experimentally in this project in combination
with studies of current literature is that the number of arms should exceed the total
number of feature-elements in the arm context vector. It was noticed that using too few
arms could mean not generating good enough variance of the elements, while using too
many arms would hurt the performance, in terms of successful recommendations, when
using small datasets.

5.4

Data extraction

The data used in the implementation is extracted from an MSSQL database containing
the entire Junkyard dataset. The extracted data that are used for training, recommendations and validation consists of three SQL- views. Each containing context of users,
items and orders respectively.

25

5.5. CONTEXT VECTORS

5.4.1

CHAPTER 5. IMPLEMENTATION

The Userdata view

The query that builds the userdata view can be viewed in the appendix, listing A.1. The
columns UserId, ZipCode, DateOfBirth and Gender in the resulting table are used when
constructing the context vector of each user (see more under 5.5.1) whereas the column
Updated is used to check when the specific user last updated his profile. Most of the
time it is used for notifying when a user changed it’s domicile i.e the ZipCode changed.
Something that is important because one of the features of a user’s context vector is
domicile.

5.4.2

The Itemdata view

The query that builds the itemdata view can be viewed in the appendix, listing A.2. The
columns ItemId, Gender and category in the resulting table are used when constructing
the context vector of each item (see more under 5.5.2) where as the columns ItemCreated,
ItemModified and CurrentStock are used to check the saldo of an item i.e if the saldo is
at zero at a specific point in time, when a recommendation is about to be made and we
know that the saldo wont increase in the near future, then obviously we do not want to
recommend that item as it would contribute to a bad user experience.

5.4.3

The Orderdata view

The query that builds the orderdata view can be viewed in the appendix, listing A.3. All
of the columns are used to train the arms of the bandit algorithm, from one point in time
to another. This view is also, however, used for validation. By chronologically observing
the data, ”future” data can be used as validation. Simplified, the algorithm looks at
which user bought what item and at what point in time. More about the training part
of the algorithm can be found under 5.6.

5.5

Context vectors

Each user, item and arm has it’s own context vector. The vectors of each arm are generated randomly, whereas the Thompson Sampling algorithm determines which arm is the
best, currently. The user and item vectors are constructed using information provided
by users upon registering, by their history of purchases and from their behaviour in the
system, whereas itemdata gets added when administrators of the system add new items
to the database.
In the following sections the context vectors are explained in more detail. The number of elements each feature is represented by are shown in Table 5.1.

26

5.5. CONTEXT VECTORS

CHAPTER 5. IMPLEMENTATION

Feature

Number of Elements

Gender

2

Category

33

Age

3

ZipCode

1

Popularity

2

Total

41

Table 5.1: Feature Data

The representation of Gender is done using one element each for male and female respectively. The Category feature has one element for each category. Age is represented
by three age spans; young − the buyer is younger than 18 years old, middle aged −
the buyer is between 18 and 33, old − where the buyer is older than 33. ZipCode is
represented by only one element consisting of the three first digits in the area-number.
The Popularity feature contains two elements; most bought last week and most bought
last month.

5.5.1

The User Context Vector

The implementation of the user context vectors follow this simple model:

Gender = Given by the user/database, can be male or female






 Category = By observing the category of purchases from a specific user
Age = Given by the user/database



ZipCode = Given by the user/database




Popularity = By observing if the user buys popular items or not

5.5.2

The Item Context Vector

The implementation of the item context vectors follow this simple model:

Gender = Given by the admin/database, can be male, female or both






 Category = Given by the admin/database
Age = By observing the age span of users who bought this item



 ZipCode = By observing the domicile of users who bought this item



Popularity = By looking at the most bought items from the last week and the last month

27

5.6. TRAINING ELEMENTS

5.5.3

CHAPTER 5. IMPLEMENTATION

The Arm Context Vector

The purpose of the context vectors of the arms is to control how much weight to put
on the different features in the context vectors of items and users when computing their
scalar product. In the initialisation phase of the algorithm, the feature-values of the
arm’s context vectors are generated randomly from a continuously uniform distribution,
to values between 0 and 1. After which they never change.

Gender = Randomly






 Category = Randomly
Age = Randomly



ZipCode = Randomly




Popularity = Randomly

generated
generated
generated
generated
generated

These vectors are needed in the Contextual Multi-Armed Bandit setting as described in
3.4.

5.6

Training elements

As mentioned in 5.1 the implemented set of algorithms are based on a content-based
filtering approach, using learning techniques for ”arms”, items and users in the setting of
Multi-Armed Bandits. The method used for the training of arms are called Thompson
Sampling and can be viewed under 5.2. The training of user profiles and items follow a
user preference model [60]. The training consists in updating the features of the context
vectors.

5.6.1

Training of Arms

The context vectors of the arms are initialised with random values as described under
5.5.3 and training the arms only consists of making sure we use the better arms more
often.

5.6.2

Training of User profiles

Assuming that a specific user is likely to buy products with equal features in the future,
user profiles progress over time and learn the features of items that are bought by the
specific user. Item features contributing to this learning process are category, gender and
popularity. The structure of the user context vector are shown in 5.5.1. User profiles get
initial training prior to running the algorithm.

28

5.7. FRAMEWORKS

5.6.3

CHAPTER 5. IMPLEMENTATION

Training of Items

Items learn the preferences of the users who purchase them over time, as shown in
5.5.2. As several of the item features rely on classifications made through learning userpreferences, items get initial training prior to running the main algorithm.

5.7

Frameworks for implementing recommender systems

As initially described in Chapter 1 this project uses frameworks for implementing recommender systems. Both of the used frameworks come with a well-documented java API,
which make them really intuitive and user-friendly. The plan, initially, was to use frameworks for redundant efforts such as implementing data structures, basic algorithms and
visual tools etc. This have been true to some extent, but in most cases implementations
have been made from scratch.

5.7.1

Apache Mahout

Apache Mahout is a scalable machine learning library and framework for implementing
recommender systems [2]. The core of the implemented application in this thesis is
written from scratch using data structures and some algorithms from Mahout; Through
use of Mahout, vectors and matrices with their algebraic operations could be used of the
shelf. Mahout also contained some built-in basic machine-learning libraries such as one
for beta distributions, used in this project.

5.7.2

LensKit

LensKit is, like Mahout, a framework for implementing recommender systems with support for some of the more common algorithms[1]. Similarly with Mahout, the implementation of this thesis only make use of some data structures from LensKit.

5.8

Implementation of Collaborative Filtering for Evaluation

As can be read in Chapter 2, the mainstream approach of implementing recommender
systems [61] to date is through use of collaborative filtering- based techniques, which
is why it makes sense to use one for comparison. The implementation of collaborative
filtering in this project is performed taking a user-based approach, meaning that similarity is computed between users rather than items. More about user-based collaborative
filtering can be found under 2.1.1. The implementation structure look like any other
Collaborative Filtering-based approach, using a similarity computation followed by a
prediction generation.

29

5.9. BASELINES

5.8.1

CHAPTER 5. IMPLEMENTATION

Similarity Computation

The standard implementations of collaborative filtering based techniques use ratings as
described in 2.1. All systems do not however make use of ratings. For instance, it is
rather trivial to realise why using ratings would not be feasible for garment-based ecommerce, as is the case for the dataset used in this thesis; the Junkyard e-commerce
platform does not make use of ratings.
Our implementation follows a model described in [62], where similarity of users are
computed using the Jaccard Distance of purchases. The Jaccard Similarity is computed
as:
|u1 ∩ u2 |
J(u1 ,u2 ) =
(5.1)
|u1 ∪ u2 |
Where, in our implementation, u1 and u2 are vectors each belonging to different users,
whose similarity is to be computed. These vectors contain Boolean values for all items in
the database, determining whether a specific item has been bought by this user or not.
Computing the Jaccard Distance as explained above will return a value 0 < J(u1 ,u2 ) < 1
and can thus be seen as ”How similar is user u1 to user u2 ?”.

5.8.2

Prediction Generation

After the similarity is computed between a user u and all other users u ∈ {u1 , u2 , . . . , un }
who bought items, the algorithm chooses which items to recommend using the following
formulae:
Sik =

X

J(uj , uk )

(5.2)

i∈uj ,j6=k

P (uk ) = top 10 Sik

(5.3)

Where the resulting Sik is the value of item i, where i is an item that user k has not
bought. J is the similarity function for a user uj and a user uk described in equation
5.1. P (uk ) is a vector containing the top ten items for user uk .

5.9

Recommender Baselines for Evaluation

Aside from the implemented, significantly different algorithms used for evaluation described in 5.8, some basic statistical models have been implemented to be used as a
kind of baselines throughout the the project. A baseline is a static algorithm for making recommendations, following some statistical model. Some examples of implemented
baselines are (strategy):
• Always recommend the most sold item, as of the last hours, days, weeks, months
or season. (People always buy the most popular items)

30

5.9. BASELINES

CHAPTER 5. IMPLEMENTATION

• Always recommend items based on gender. (Men and women always buy the same
items as other men and women, respectively)
• Always recommend items based on age. (People at a certain age always buy same
items as other people in the same age)
• Always recommend items based on domicile. (People from city X always buy the
same items as other people from city X)
Baselines have been used before any other extensive evaluation method, something that
has been of great help when implementing the application, as these baselines can be seen
as evaluation in its most basic form.

31

6
Result
ulti-Armed Bandit Algorithms are rarely used in recommender systems
for e-commerce. In this chapter follows a presentation of the results achieved
in this project using our own implementation of The Contextual MultiArmed Bandit Algorithm applied to the a dataset from a live e-commerce
system. This chapter will focus on the actual results. For evaluation and discussion
regarding how good the results were, see Chapter 7.

M
6.1

Recommender System

Taking a privacy-preserving approach, recommendations performed by the implemented
application can be split into three categories or levels of context which can be seen under
6.1.1, 6.1.2 and 6.1.3. In the following series of plots, the results of the implemented
application using Contextual Multi-Armed Bandits are illustrated. Each subsection contains four graphs illustrating different time periods and for each subsection the data is
limited taking a more privacy-preserving approach. The graphs show a percentage of
correct predictions on the y-axis and the total number of predictions made on the x-axis.

6.1.1

Using all available context

The lowest level of privacy consist in using all available context, meaning that the algorithm takes everything from user-submitted data, to observed and learned user-behaviour
and -history, into account when predicting purchases. Here follow some graphs with different time aspects, visualising the performance with number of predictions on the x-axis
and correctness in percent on the y-axis using this policy.
Graphs in figure 6.1 show all arms relative to the best arm. The best arm is chosen
as the one with best performance posterior of the simulation. Notice that ”All arms”
follow the same pattern as the ”Best arm” towards the end. This is due to the fact
32

6.1. RECOMMENDER SYSTEM

CHAPTER 6. RESULT

that the algorithm is more or less done with the exploration phase and has started the
exploiting phase, choosing the best arm over and over.

Predictions made of data from 1 week in time

Predictions made of data from 1 month in time
25

Correct predictions (%)

Correct predictions (%)

30

25

20

15
Best Arm
All Arms

10

0

1000

2000

3000

4000

5000

6000

20

15
Best Arm
All Arms

10

7000

0

0.5

Number of predictions
Predictions made of data from 3 months in time

2
×104

25

Correct predictions (%)

Correct predictions (%)

1.5

Predictions made of data from 6 months in time

30
25
20
15
10
5

1

Number of predictions

Best Arm
All Arms

0

2

4

6

Number of predictions

8

15

10
Best Arm
All Arms

5

10
×10

20

4

0

0.5

1

Number of predictions

Figure 6.1: Results using all available context

33

1.5

2
×105

6.1. RECOMMENDER SYSTEM

6.1.2

CHAPTER 6. RESULT

Using only user-submitted data

The middle level of privacy consist in only using data submitted by the users of a system. Data that they agreed to submit upon registering an account or purchase in the
system. Recommendations are done using only this data, without having the algorithm
train on observed user behaviour in the system. Here follow some graphs with different
time aspects, visualising the performance with number of predictions on the x-axis and
correctness in percent on the y-axis using this policy.

Predictions made of data from 1 week in time

Predictions made of data from 1 month in time
14

Correct predictions (%)

Correct predictions (%)

15

10

5
Best Arm
All Arms

0

0

1000

2000

3000

4000

5000

6000

12
10
8
6
4
2

7000

Best Arm
All Arms

0

0.5

Number of predictions
Predictions made of data from 3 months in time

2
×104

15

Correct predictions (%)

Correct predictions (%)

1.5

Predictions made of data from 6 months in time

15

10

5
Best Arm
All Arms

0

1

Number of predictions

0

2

4

6

Number of predictions

8

10

5
Best Arm
All Arms

0

10
×104

0

0.5

1

Number of predictions

Figure 6.2: Results using only user-submitted context.

34

1.5

2
×105

6.2. THE APPLICATION

6.1.3

CHAPTER 6. RESULT

Using anonymous data

The highest level of privacy consist in total anonymous data. Recommendations are
generated only through observing the flow of products, what are being bought and
when.Due to the nature of Contextual Multi-Armed Bandit Algorithms, the scenario of
anonymity can only be applied to a certain extent with the dataset used in this thesis.
One way of making anonymous recommendations can be seen as recommending the most
popular items from a certain time period. Results of this can be seen under 6.3.1.

6.2

The Application

The implemented application is implemented in a scalable manor but without any graphical interfaces, meaning that the core of the application is easily modified to use more or
less features, arms, training sets, predictions etc. Inputs are accepted following a simple
model, as can be seen in 5.4, meaning input could be taken from other systems with
some slight modifications. The application also includes support for automatic logging
and plotting of the results. Everything to make the application as easily understandable
and usable as possible.

6.3

Algorithms for Evaluation

The implemented algorithms for evaluation and comparison consist in two different approaches. The first one is through use of simple statistical models, and the second one
is a full-scale implementation of a User-Based Collaborative Filtering Algorithm.

6.3.1

Simple Statistical Models

Simple statistical models have been implemented, as described under 5.9, and used as
baselines while implementing the main application, something that have been of great
help when trying new methods early on in the implementation steps. It is however
pointless to use baselines with poor performance in the final result, which is why the
following graphs only show the baseline with the best performance. This baseline is a
statistical model based on the purchases from the last week of any point in time. It
will always recommend the ten most bought items as of the last week, to all users. The
following graphs got the same layout as the ones in 6.1, namely with a percentage of
correct recommendations on the y-axis and the total number of recommendations made
on the x-axis.

35

6.3. ALGORITHMS FOR EVALUATION

CHAPTER 6. RESULT

Predictions made of data from 1 week in time

Predictions made of data from 1 month in time
18

Correct predictions (%)

Correct predictions (%)

17
16
15
14
13
12

16

14

12

Popularity(Baseline)

11

0

1000

2000

3000

4000

5000

6000

Popularity(Baseline)

10

7000

0

0.5

Number of predictions
Predictions made of data from 3 months in time

2
×104

Correct predictions (%)

18

16

14

12

16

14

12

Popularity(Baseline)

10

1.5

Predictions made of data from 6 months in time

18

Correct predictions (%)

1

Number of predictions

0

2

4

6

Number of predictions

8

Popularity(Baseline)

10

10
×104

0

0.5

1

1.5

Number of predictions

Figure 6.3: Graphs showing recommendations based solely on popularity following the
baseline that recommends the most bought items as of the last week

36

2
×105

6.3. ALGORITHMS FOR EVALUATION

6.3.2

CHAPTER 6. RESULT

User-Based Collaborative Filtering

A full-scale implementation of an application using collaborative filtering-based algorithms have been implemented as described under 5.8. Here follow some graphs of its
performance where the graphs got the same layout as the ones in 6.1 and 6.3.1, namely
with a percentage of correct recommendations on the y-axis and the total number of
recommendations made on the x-axis.

Predictions made of data from 1 week in time

Predictions made of data from 1 month in time
6

Correct predictions (%)

Correct predictions (%)

7
6
5
4
3

5
4
3
2
1

Jaccard Similarity

2

0

1000

2000

3000

4000

5000

6000

Jaccard Similarity

0

7000

0

0.5

Number of predictions
Predictions made of data from 3 months in time

2
×104

Correct predictions (%)

6

5
4
3
2
1

5
4
3
2
1

Jaccard Similarity

0

1.5

Predictions made of data from 6 months in time

6

Correct predictions (%)

1

Number of predictions

0

2

4

6

Number of predictions

8

Jaccard Similarity

0

10
×104

0

0.5

1

1.5

Number of predictions

Figure 6.4: Graphs showing result when recommending items based on the jaccard similarity between users.

37

2
×105

7
Discussion
s stated in the introduction, this thesis aims to evaluate the performance
of Multi-Armed Bandit Algorithms in an e-commerce environment. In this
chapter evaluation of user-experience and comparisons with other algorithms
will be presented together with thoughts regarding the result and future work.

A
7.1

Evaluating The Result

The result can be evaluated depending on several different aspects. In this section focus
will be put on evaluation of performance and user-experience as well as some thoughts
regarding why the result turned out the way it did.

7.1.1

Performance

The percentage of correct predictions can be seen on the y-axis of the graphs in section
6.1. By comparing the graphs in figure 6.1 one can draw the conclusion that the best
recommendations are made in the time interval between one week to one month posterior to starting the application. Predictions in this period got an accuracy of 19 to 21
percent. Seeing how this is in a garment-based e-commerce environment it makes sense
that new collections of products are added every other month, whereas the products from
last month’s collection do not sell as good. The simulation made in these graphs starts
in late spring, 2012-05-01, meaning that recommendations of cloths for the summer are
more likely to be of satisfactory than a few months later when summer-based cloths are
recommended for the fall or winter. The conclusion here is that if this application was
to run in a live system, it would make sense to reset the trained data prior to releasing
a new collection and/or before each new season.
As mentioned earlier, the dataset used in this thesis contains only static data. Meaning
that for a prediction to be seen as correct, a recommended item has to be purchased at
38

7.1. EVALUATION

CHAPTER 7. DISCUSSION

some point in time after the recommendation is performed. The problem using static
data rather than testing an application live in this setting is that possibly good recommendations are not measured, because an item was not purchased. One can never know
if the user would have bought more products if products that she could possibly like
would be presented to her. Hence the only thing that can be measured is actual purchases, while in theory a user is likely to buy more products based on the recommended
items. Something that could be said to be certain about the application here is that it
predicts correct items with at least 19-21 percent accuracy and probably higher in most
cases.
All the graphs in Chapter 6 expressing anonymous data to some extent, proves that
taking a more privacy-preserving approach, actually gives worse performance in terms
of measuring successful recommendations. By comparing the graphs in figure 6.1 where
the algorithm is using all available context, with the other graphs in figure 6.2 where
the algorithm is using less context, it is possible to see a difference of about 6-10 percent
depending on which graphs are observed. It is hard to tell if it is worth it or not, but
seeing how things work in reality where most people accept any license agreement of applications or web-pages, one can draw the conclusion that as long as the observed data
is used in ways that benefit the actual user and not a third party, it is at least socially
acceptable. Facebook is a good example of this, which in their licence agreement [63]
state that they have all the rights to any data you submit or take part of online, when
using their services. They also state that your data will be accessible by companies and
third parties of their choosing.
There are also other aspects to a good result than through only measuring the number of
successful recommendations, such as user experience and satisfaction level of customers.
Unfortunately these are hard to measure without deploying the system live. See more
about user-experience under 7.1.2.
So far we have seen that the implemented application can predict shopping behaviour
of most users with a probability of around 20 percent. So the question is how well
does other similar applications perform on similar datasets. One similar application in
a similar setting can be seen in [61] where they use significantly different algorithms and
achieve correct recommendations with a probability of about 17 percent. The authors
of [61] make use of a collaborative filtering-based techniques, but with significantly better performance than the collaborative filtering-based algorithms used in this project.
Nevertheless, the content-based filtering implementation using Contextual Multi-Armed
Bandits in this thesis outperforms the algorithms used in [61], something that ought to
be considered as successful. In figure 7.1 below, the different implemented algorithms
of this thesis is presented with different colours, using the time interval of one month
because of its relevance as described above.

39

7.1. EVALUATION

CHAPTER 7. DISCUSSION

Predictions made of data from 1 month in time
25
Best Arm
All Arms
Popularity(Baseline)
Jaccard Similarity

Correct predictions (%)

20

15

10

5

0

0

2000

4000

6000

8000

10000

12000

14000

16000

Number of predictions

Figure 7.1: Graph showing the results of all implemented algorithms from one month in
time

And as can be seen in this graph, the Contextual Multi-Armed Bandit Algorithm outperforms all of the others. Notable is also that the popularity baseline algorithm appears
to have pretty high performance. Something that is easily realised why it is the case.
This strategy is very bad in other aspects as if taking User-Experience into account. See
section 7.1.2 for more details regarding this.
The next graph, in figure 7.2, shows the performance of all algorithms in different colours,
measuring performance of purchases over one year.

40

18000

7.1. EVALUATION

CHAPTER 7. DISCUSSION

Predictions made of data from 12 months in time
30
Best Arm
All Arms
Popularity(Baseline)
Jaccard Similarity

Correct predictions (%)

25

20

15

10

5

0

0

0.5

1

1.5

2

2.5

3

Number of predictions

3.5
×105

Figure 7.2: Graph showing the results of all implemented algorithms from one year in time

Here one can see that the result does not change even in the long run. All algorithms
perform best during the first months due to reasons expressed in the beginning of this
section, but even after a long time the order of good versus bad algorithms do not change.
Something notable is what happens in the time span between 1 and 3 months, where
we can see a dip in performance in terms of successful recommendations. The graph
stretches from 2012-05-01 to 2012-07-31. In this time period there is a shift in products
wanted by the users. A user that has bought for instance some pairs of jeans during
May has a high probability of being recommended jeans during the later months too.
However, during the summer it is more likely that the user wants to buy shorts and other
products more suitable for use during the summer. As can be seen in graph 7.2, the popularity curve also become worse during these months. This means that user-behaviour
is harder to predict during these months. This might be the case either because there is
a larger set of items that can be recommended, or because user behaviour is not as predictable during the summer as during the spring. As mentioned initially in this chapter,
it is also likely that new collections of garment are released before each season, which
might be another contributing factor to the dip in performance during the specific time
spans.
One could wonder: why did the collaborative filtering-based algorithm using Jaccard

41

7.1. EVALUATION

CHAPTER 7. DISCUSSION

Similarity perform so poorly with the Junkyard dataset, in comparison with the other
algorithms in this thesis and the ones described in [61]. This, of course, might depend on
a variety of different reasons such as algorithms or datasets being significantly different.
Seeing how the focus of this evaluation is on the performance of Multi-Armed Bandits,
further investigation of the poor result of the above mentioned will be left for future work.
As explained in 3.2, regret bounds are often used for measuring negative emotion experience in decision theory and probabilistic modelling. In figure 7.3 below, the total regret
is illustrated for ’All Arms’, ’Most Buys’ and ’Jaccard Similarity’ relative to the ’Best
Arm’.
Regret of All Arms, Popularity and Jaccard Similarity relative to the Best Arm

100

All Arms
Popularity(Baseline)
Jaccard Similarity

90

80

Regret (%)

70

60

50

40

30

20

10

0

2000

4000

6000

8000

10000

12000

14000

16000

Number of predictions

Figure 7.3: Regret of ’All Arms’, ’Most Buys’ and ’Jaccard Similarity’ relative to the Best
Arm.

An interesting scenario that was stumbled upon is if the following happens (7.4):

42

18000

7.1. EVALUATION

CHAPTER 7. DISCUSSION

Predictions made of data from 1 month in time
20
Best Arm
All Arms

18
16

Correct predictions (%)

14
12
10
8
6
4
2
0

0

1000

2000

3000

4000

5000

6000

Number of predictions

Figure 7.4: All Arms outperforming Best Arm for some time

What can be seen here is running the algorithm for six months, and where All Arms
seemingly perform better than the Best Arm at some point in time. There is a simple
explanation to this phenomenon. Posterior to running the algorithm, the algorithm looks
at all arms to see which one that is having the best performance. Thus the arm with
the best performance posterior to running the algorithm is not necessarily the best arm
throughout the whole simulation.

7.1.2

User-Experience

User-Experience is more tricky to measure than performance. Despite suggesting items
that are likely to be bought statistically, it does not mean that it will make a specific
customer happy. Recommending items that might be of no value for a specific customer,
but are best-sellers, is unarguably not the best way to present recommendations even
if it will show good performance. This can be seen in the graphs in figure 6.3 where
recommendations are made entirely based on whats the currently most bought items as
of the last week.

43

7000

7.1. EVALUATION

CHAPTER 7. DISCUSSION

Like a physical store
When visiting a physical store, you would not want the shopping assistant to only propose the same items as she is proposing to everyone else. For instance by proposing only
the best-sellers as of the last week, not taking anything else into consideration. Instead
you would want the shopping assistant to listen to what kind of items you like and have
enjoyed earlier, and what you are currently looking for. The same thing applies for most
people when browsing E-Commerce websites or platforms. If you browse for or buy certain items, the system should adapt to this and actually try to help you find what you
might be looking for in the future. That is if you are looking to buy a pair of shoes and
you are recommended ten different shirts, just because the shirts happened to be on sale
last week meaning that a lot of people have purchased them, you will not be satisfied or
any closer to finding the pair of shoes you were after.
If however, let us say, the shopping assistant was to follow you around noting what
you were looking at or even what you were buying in other stores, you would probably
find her creepy and probably stay away from her stores. This despite the fact that it
might mean ending up with an item you enjoy, faster. Once again, the same idea can
be applied in E-Commerce systems; using all data you can from a specific user is not
ethically acceptable. It is however more socially acceptable as described in 7.1.1. This
could be data such as cookie-based history of other browsed web sites. It could also be
data retrieved from buying and selling information gathered in other systems.
Our implementation
When implementing the algorithm in this project, the physical store was kept in mind.
Users should not relate to the system as the creepy shopping assistant. The implemented
application thus only focus on using data gathered from the platform which it was built
for and even there discussing different privacy-preserving aspects. Neither should users
relate to the system as the lazy shopping assistant who only gives the same recommendations to everyone. Instead focus was put on making the customers feel that the system or
shopping assistant actually proposed valid items that the customer could be interested
in. This by giving ten recommendations that match the specific customer’s profile.
Taking it live
Observing the shopping history of a user in the store is an intuitively good technique to
figure out what the user is interested in. However, if taking the system live,predictions
would also based on the current live session. The taste of a user would probably most
often be similar from the different times she visits the website, but she might want to
shop from different categories during different sessions. Looking for shoes at one time
should not mean that the user only gets shoe recommendations the second time she
browses the site. If the user however only looks for gender-specific items, most of the
time there would be no point in suggesting the opposite.

44

7.2. FUTURE WORK

CHAPTER 7. DISCUSSION

Best-seller recommendations are not useless in any way. Using this application in a
live setting could mean using best-seller recommendations on the front page so that
people occasionally browsing the page in hopes of finding something they like might get
lucky, while the personalised recommendations could be used on the specific user-pages.

7.2

Future Work

There are still issues related with the result of this project that could be desirable to
address, but that goes out of the scope of this thesis. In this section we will present
some concrete ideas of possible future work related to this thesis.

7.2.1

Synthetic Data

Synthetic data is, as opposed to authentic data, generated within some behavioural
model. It is explained in [64] and described as: ”Synthetic data can be defined as data
that are generated by simulated users in a simulated system, performing simulated actions.
One obvious use-area for synthetic data is the possibility of doing realistic testing of
a system or parts of a system before deploying them live. There are however many difficulties that need to be considered before implementation can start. Difficulties that are
not entirely addressed in current research as it varies in different settings and systems.

7.2.2

Extensive Collaborative Filtering-Based Algorithm

The implemented collaborative filtering-based algorithm in this thesis perform poorly in
comparison to what current research expresses when describing collaborative filteringbased algorithms. It would be interesting to make use of different similarity computing
techniques to see if it could, first of all, provide a better result than the implemented
collaborative filtering-based algorithm in this thesis, but also if it could beat the performance of the Contextual Multi-Armed Bandit Algorithm implemented in this thesis.

45

8
Conclusion
n this thesis the performance of Contextual Multi-Armed Bandit Algorithms have
been measured and evaluated in terms of successful recommendations, user-experience
and possibility of working using privacy-preserving methods, in a garment-based
e-commerce environment. The result can be seen in chapter 6. In the aspect of making successful recommendations and giving a good user-experience the results look good.
But also in the cases where privacy-preserving approaches have been taken, the results
are fairly good. Depending on the setting and system, if required, it would thereby be
feasible to consider even the privacy-preserving modes of the algorithm as performance
in terms of successful recommendations are fairly good despite those restrictions.

I

8.1

The Research Question

In the introduction we defined the research question as:
Would the algorithmic approach of Contextual Multi-Armed Bandit Algorithms perform
well enough to belong among the different algorithmic approaches to consider when implementing recommender systems?
A justified answer to this question, with respect to the results presented in Chapter 6 and
the reasoning of the results presented in Chapter 7, is ”yes”. A reasonable conclusion that
can be made from observing the result where purchasing behaviour is predicted with a
probability of over 20 % is that considering Contextual Multi-Armed Bandit Algorithms
when implementing recommender systems is highly appropriate.
The more complex but also most suitable answer to the research question is ”probably”
or ”it depends on the setting and environment”. As the name Contextual Multi-Armed
Bandit Algorithms indicates, the algorithm makes use of contexts such as a user’s context, personal information, and might thus not be eligible in systems where all data is to
be totally anonymous or where only non-personal data exists. However, as can be seen

46

8.1. THE RESEARCH QUESTION

CHAPTER 8. CONCLUSION

in Chapter 6 results are fairly good despite using less user context, but there might be
other algorithms more suitable for this purpose.
Another conclusion that can be made using Contextual Multi-Armed Bandit Algorithms
in garment-based e-commerce systems, is when the e-commerce system does not make
use of ratings, where otherwise reliable collaborative filtering-based algorithms might
not perform so well (see 6.3.2), Contextual Multi-Armed Bandit Algorithms could work
better because it observes user-context and -behaviour rather than ratings. Although
this thesis is only testing Contextual Multi-Armed Bandit Algorithms using static data
from one garment-based e-commerce system, which does not make use of ratings. To
be certain that equally good results can be achieved frequently, it would be necessary
to test the algorithm on data from more and different systems. Especially significantly
different systems, such as ones that does not involve clothing.
Contextual Multi-Armed Bandit Algorithms work in systems with or without ratings.
As long as there exist recorded data of users and their behaviour in the system, Contextual Multi-Armed Bandit Algorithms should have good chances of providing good
and reliable recommendations in the aspect of user-experience as well as of performing
successful recommendations.





