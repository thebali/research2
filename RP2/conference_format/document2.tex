%%% Note that the a4paper option is mainly intended so that authors in
% countries using A4 can easily print to A4 and see how their papers will
% look in print - the typesetting of the document will not typically be
% affected with changes in paper size (but the bottom and side margins will).
% Use the testflow package mentioned above to verify correct handling of
% both paper sizes by the user's LaTeX system.
%
% Also note that the "draftcls" or "draftclsnofoot", not "draft", option
% should be used if it is desired that the figures are to be displayed in
% draft mode.
%
\documentclass[11pt,conference]{IEEEtran}
% Add the compsoc option for Computer Society conferences.
%
% If IEEEtran.cls has not been installed into the LaTeX system files,
% manually specify the path to it like:
% \documentclass[conference]{../sty/IEEEtran}





% Some very useful LaTeX packages include:
% (uncomment the ones you want to load)

% *** MATH PACKAGES ***
%
\usepackage[cmex10]{amsmath}
% A popular package from the American Mathematical Society that provides
% many useful and powerful commands for dealing with mathematics. If using
% it, be sure to load this package with the cmex10 option to ensure that
% only type 1 fonts will utilized at all point sizes. Without this option,
% it is possible that some math symbols, particularly those within
% footnotes, will be rendered in bitmap form which will result in a
% document that can not be IEEE Xplore compliant!
%
% Also, note that the amsmath package sets \interdisplaylinepenalty to 10000
% thus preventing page breaks from occurring within multiline equations. Use:
%\interdisplaylinepenalty=2500
% after loading amsmath to restore such page breaks as IEEEtran.cls normally
% does. amsmath.sty is already installed on most LaTeX systems. The latest
% version and documentation can be obtained at:
% http://www.ctan.org/tex-archive/macros/latex/required/amslatex/math/





% *** SPECIALIZED LIST PACKAGES ***
%
%\usepackage{algorithmic}
% algorithmic.sty was written by Peter Williams and Rogerio Brito.
% This package provides an algorithmic environment fo describing algorithms.
% You can use the algorithmic environment in-text or within a figure
% environment to provide for a floating algorithm. Do NOT use the algorithm
% floating environment provided by algorithm.sty (by the same authors) or
% algorithm2e.sty (by Christophe Fiorio) as IEEE does not use dedicated
% algorithm float types and packages that provide these will not provide
% correct IEEE style captions. The latest version and documentation of
% algorithmic.sty can be obtained at:
% http://www.ctan.org/tex-archive/macros/latex/contrib/algorithms/
% There is also a support site at:
% http://algorithms.berlios.de/index.html
% Also of interest may be the (relatively newer and more customizable)
% algorithmicx.sty package by Szasz Janos:
% http://www.ctan.org/tex-archive/macros/latex/contrib/algorithmicx/




%\usepackage{array}




% *** PDF, URL AND HYPERLINK PACKAGES ***
%
%\usepackage{url}
% url.sty was written by Donald Arseneau. It provides better support for
% handling and breaking URLs. url.sty is already installed on most LaTeX
% systems. The latest version and documentation can be obtained at:
% http://www.ctan.org/tex-archive/macros/latex/contrib/url/
% Basically, \url{my_url_here}.




% *** Do not adjust lengths that control margins, column widths, etc. ***
% *** Do not use packages that alter fonts (such as pslatex).         ***
% There should be no need to do such things with IEEEtran.cls V1.6 and later.
% (Unless specifically asked to do so by the journal or conference you plan
% to submit to, of course. )


% correct bad hyphenation here
\hyphenation{op-tical net-works semi-conduc-tor}

\usepackage[english]{babel}
\usepackage{graphicx, type1cm, lettrine}
\usepackage{cite}


\begin{document}
%
% paper title
% can use linebreaks \\ within to get better formatting as desired
% Do not put math or special symbols in the title.
% title actual
% Collaborative filtering: Techniques and Applications
\title{Generating Recommendations from SVD, Neural network, kNN}

% author names and affiliations
% use a multiple column layout for up to three different
% affiliations


\author{
\IEEEauthorblockN{Rahul Bali}
\IEEEauthorblockA{Department of Computer Science\\
Punjab Engineering College\\
Chandigarh, India\\
Email: rahulrdb18@gmail.com}
% email: rahulbali.mecse16@pec.edu.in
\and
\IEEEauthorblockN{Shilpa Verma}
\IEEEauthorblockA{Department of Computer Science\\
Punjab Engineering College\\
Chandigarh, India\\
Email: shilpaverma.pec@gmail.com}}



% conference papers do not typically use \thanks and this command
% is locked out in conference mode. If really needed, such as for
% the acknowledgment of grants, issue a \IEEEoverridecommandlockouts
% after \documentclass

% for over three affiliations, or if they all won't fit within the width
% of the page, use this alternative format:
% 

% a kind of multiline comment
\iffalse

\author{\IEEEauthorblockN{Michael Shell\IEEEauthorrefmark{1},
Homer Simpson\IEEEauthorrefmark{2},
James Kirk\IEEEauthorrefmark{3}, 
Montgomery Scott\IEEEauthorrefmark{3} and
Eldon Tyrell\IEEEauthorrefmark{4}}
\IEEEauthorblockA{\IEEEauthorrefmark{1}School of Electrical and Computer Engineering\\
Georgia Institute of Technology,
Atlanta, Georgia 30332--0250\\ Email: see http://www.michaelshell.org/contact.html}
\IEEEauthorblockA{\IEEEauthorrefmark{2}Twentieth Century Fox, Springfield, USA\\
Email: homer@thesimpsons.com}
\IEEEauthorblockA{\IEEEauthorrefmark{3}Starfleet Academy, San Francisco, California 96678-2391\\
Telephone: (800) 555--1212, Fax: (888) 555--1212}
\IEEEauthorblockA{\IEEEauthorrefmark{4}Tyrell Inc., 123 Replicant Street, Los Angeles, California 90210--4321}}

I don't want this to happen
\fi


% make the title area
\maketitle

% As a general rule, do not put math, special symbols or citations
% in the abstract
\begin{abstract}
	The world has seen much change in ways which recommendations were made since the first arrival of the collaborative filtering in providing recommendations. The paper below describes the various advancements in the featured alorithms discussed here. The state of the art recommendations engines have been described with their best chances of survival. Certain trends have also made it to this work.
\end{abstract}


% no keywords
% For peer review papers, you can put extra information on the cover
% page as needed:
% \ifCLASSOPTIONpeerreview
% \begin{center} \bfseries EDICS Category: 3-BBND \end{center}
% \fi
%
% For peerreview papers, this IEEEtran command inserts a page break and
% creates the second title. It will be ignored for other modes.
\IEEEpeerreviewmaketitle


%% keywords for the writing.
%%% information overload
%%% computer-human interaction
%%% 
\section{Introduction}

\cite{linden2003amazon}
\cite{sarwar2001item}
\cite{su2009survey}
\cite{zuva2012survey}
\cite{ekstrand2011collaborative}
\cite{herlocker2004evaluating}
\cite{linden2003amazon}
\cite{adomavicius2005toward}
\cite{park2012literature}


\section{Recommender Systems Techniques}

\section{Conclusion}

\newpage
\bibliographystyle{IEEEtran}
\bibliography{IEEEfull}

% that's all folks
\end{document}


